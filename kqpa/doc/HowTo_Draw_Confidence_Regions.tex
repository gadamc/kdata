\documentclass[10pt, a4paper]{article}
\usepackage[english]{babel}
\usepackage[utf8]{inputenc}
\usepackage[T1]{fontenc}
\usepackage{palatino}
\usepackage{geometry}
\geometry{hmargin=2cm,top=2cm,bottom=2cm}


\title{HowTo_Draw_Confidence_Regions}
\author{Daniel Wegner}

\begin{document}


In order to draw contour lines representing 90\% confidence regions for single EDW events one has to do the following steps:

\begin{enumerate}
\item Load the kqpa library.
\begin{verbatim} gSystem->Load("$KDATA_ROOT/lib/libkqpa.so"); \end{verbatim}
\item Make a KQContourPointList object.
\begin{verbatim} KQContourPointList aList; \end{verbatim}
\item Fill the list with points (Q, ERecoil, $\sigma_{E_{ion}}$ , $\sigma_{E_{heat}}$)
\begin{enumerate}
\item Read an ASCII file in the current working directory with lines of the form <$Q$> <$E_{recoil}$> <$\sigma_{E_{ion}}$> <$\sigma_{E_{heat}}$> 
\begin{verbatim} aList.ReadASCIIFile(aFileName); \end{verbatim}
\item Add events manually
\begin{verbatim} aList.AddPoint(aQvalue,anEnergyRecoil,aSigmaIon,aSigmaHeat); \end{verbatim}
\end{enumerate}
\item Draw the events in an empty frame
\begin{verbatim} aList.Draw(anOption = ""); \end{verbatim}
\end{enumerate}

The default values of the empty frame ($E_{recoil} = 0..1000 keV$, $Q = 0..2$) can be changed with set methods
\begin{verbatim}
aList.SetQvalueMax(aNewQvalueMin);
aList.SetQvalueMin(aNewQvalueMin);
aList.SetEnergyRecoilMax(aNewEnergyRecoilMax);
aList.SetEnergyRecoilMin(aNewEnergyRecoilMin);
\end{verbatim}

The list of points can be cleared by
\begin{verbatim}
aList.ClearPoints();
\end{verbatim}
and single events can be removed by
\begin{verbatim}
aListRemovePoint(anIndex);
\end{verbatim}
In order to find valid indices the size of the list can be retrieved by
\begin{verbatim}
UInt_t aSize = aList.GetEntries();
\end{verbatim}

Also single events can be created and drawn:
\begin{verbatim}
KQContourPoint anEvent(aQvalue,
                       anEnergyRecoil,
                       aSigmaIon,
                       aSigmaHeat,
                       aSigmaIonHeat,
                       aConfidenceLevel,
                       aVoltageBias,
                       anEpsilon);
anEvent.Draw(anOption="");
\end{verbatim}
In case that there might be changes necessary on the empty frame or the contour function, they can directly be retrieved by
\begin{verbatim}
TF2* aFunction = aList.GetEmptyFrame();
TF2* aFunction = anEvent.GetFunction();
\end{verbatim}











 


\end{document}