\section{Derivation}
The aim of this paper is to derive the propability density of the quantity
\begin{gather}
\vec{q} = \begin{pmatrix} E_{recoil} \\ Q \end{pmatrix}
\end{gather}
The quantity $\vec{q}$ depends from 
\begin{gather}
\vec{r} = \begin{pmatrix} E_{ion} \\ E_{heat} \end{pmatrix}
\end{gather}
The dependency is given by
\begin{gather}
T: \vec{r} \rightarrow \vec{q} , \begin{pmatrix} E_{ion} \\ E_{heat} \end{pmatrix}  \rightarrow \begin{pmatrix} E_{recoil} \\ Q \end{pmatrix} = \begin{pmatrix} \left( 1 + \frac{V}{\epsilon_\gamma} \right) E_{heat} - \frac{V}{\epsilon_\gamma} E_{ion} \\ \frac{E_{ion}}{\left( 1 + \frac{V}{\epsilon_\gamma} \right) E_{heat} - \frac{V}{\epsilon_\gamma} E_{ion}} \end{pmatrix}
\end{gather}
Then the inverse transformation is given by
\begin{gather}
T^{-1}: \vec{q} \rightarrow \vec{r}, \begin{pmatrix} E_{recoil} \\ Q \end{pmatrix} \rightarrow \begin{pmatrix} E_{ion} \\ E_{heat} \end{pmatrix} = \begin{pmatrix} Q E_{recoil} \\ \frac{1 + Q \frac{V}{\epsilon_\gamma}}{1 + \frac{V}{\epsilon_\gamma}} E_{recoil} \end{pmatrix}
\end{gather}
The derivative then is given by
\begin{gather}
T'(\vec{r}) = \begin{pmatrix} \frac{\partial E_{recoil}}{\partial E_{ion}} & \frac{\partial E_{recoil}}{\partial E_{heat}} \\ \frac{\partial Q}{\partial E_{ion}} & \frac{\partial Q}{\partial E_{heat}} \end{pmatrix}
= \begin{pmatrix} - \frac{V}{\epsilon_\gamma} & 1 + \frac{V}{\epsilon_\gamma} \\ \frac{1}{E_{recoil}} + \frac{V}{\epsilon_\gamma} \frac{E_{ion}}{E_{recoil}^2} & - \left( 1 + \frac{V}{\epsilon_\gamma} \right) \frac{E_{ion}}{E_{recoil}^2} \end{pmatrix}
\end{gather}
and thus the determinant of it is
\begin{gather}
\det T'(\vec{r}) =  \frac{V}{\epsilon_\gamma} \left( 1 + \frac{V}{\epsilon_\gamma} \right) \frac{E_{ion}}{E_{recoil}^2} - \left(1 + \frac{V}{\epsilon_\gamma}\right) \left( \frac{1}{E_{recoil}} + \frac{V}{\epsilon_\gamma} \frac{E_{ion}}{E_{recoil}^2} \right) = - \frac{1 + \frac{V}{\epsilon_\gamma}}{E_{recoil}}
\end{gather}
Obviously for all values for $E_{recoil}$ the determinant doesn't vanish and thus the transformation is invertible in the whole domain according to the inverse function theorem.

If we assume that $\vec{r}$ is a multivariate normal-distributed quantity, that means it follows the propability density function
\begin{gather}
f(\vec{r}) = \frac{1}{2 \pi \sqrt{\det C}} \exp \left( - \frac{1}{2} (\vec{r} - \vec{r}_0)^T C^{-1} (\vec{r} - \vec{r}_0) \right) \label{multigaus}
\end{gather}
with the covariance matrix
\begin{gather}
C = \begin{pmatrix} \sigma_{ion}^2 & \sigma_{ion-heat}^2 \\ \sigma_{ion-heat}^2 & \sigma_{heat}^2 \end{pmatrix}
\end{gather}
, we get the propability density function $g(\vec{q})$ \cite{Henze}[p. 246]
\begin{gather}
g(\vec{q}) = \frac{f(T^{-1}(\vec{q}))}{\left| det T'(T^{-1}(\vec{q})) \right|} \\ =  \exp \left( - \frac{1}{2}  \begin{pmatrix} Q E_{recoil} - \overline{E_{ion}}\\ \frac{1 + Q \frac{V}{\epsilon_\gamma}}{1 + \frac{V}{\epsilon_\gamma}} E_{recoil} - \overline{E_{heat}} \end{pmatrix}^T \begin{pmatrix} \sigma_{ion}^2 & \sigma_{ion-heat}^2 \\ \sigma_{ion-heat}^2 & \sigma_{heat}^2 \end{pmatrix}^{-1} \begin{pmatrix} Q E_{recoil} - \overline{E_{ion}} \\ \frac{1 + Q \frac{V}{\epsilon_\gamma}}{1 + \frac{V}{\epsilon_\gamma}} E_{recoil} - \overline{E_{heat}} \end{pmatrix} \right) \\
\times  \frac{\left| E_{recoil} \right| }{2 \pi \sqrt{\sigma_{ion}^2 \sigma_{heat}^2 - \sigma_{ion-heat}^4} \left(1 + \frac{V}{\epsilon_\gamma}\right)} \label{propfunc}
\end{gather}