\subsection{script}
The plots have been created by the script on in the scripts directory:
\begin{verbatim}
$KDATA_ROOT/kqpa/scripts/ERecoilQDist_v30.C
\end{verbatim}
This file offers some methods:
\begin{enumerate}
\item \begin{verbatim}
ERecoilQDist_v30(Double_t anEIonMean = 100,
                 Double_t anEHeatMean = 100,
                 Double_t anEIonSigma = 1,
                 Double_t anEHeatSigma = 1,
                 Double_t aNumBinsX = 2000,
                 Double_t aNumBinsY = 2000,
                 Long_t aNumEntries = 1E9,
                 Double_t aV = 3,
                 Double_t anEpsilon = 1,
                 Option_t* aFitOption = "`0LI"')
\end{verbatim}
This methods creates two histograms "`mchist"' from creating random numbers distributed according the given parameters The histograms have the dimensions (aNumBinsX,aNumBinsY) and the boundaries are chosen, so that it covers $\pm$ aNumSigmas standard deviations calculated from error propagation around the center value.
\begin{gather}
\overline{E_{Recoil}} = \left(1 + \frac{V}{\epsilon_\gamma} \right) \overline{E_{Heat}} - \frac{V}{\epsilon_\gamma} \overline{E_{Ion}} \\
\overline{Q} = \frac{\overline{E_{Ion}}}{\overline{E_{Recoil}}} 
\end{gather}
In the case of $E_{Recoil}$ this is the expectation value
\begin{gather}
\mean{E_{Recoil}} = \overline{E_{recoil}} \\
\mean{E_{Ion}} = \overline{E_{Ion}} \\
\mean{E_{Heat}} = \overline{E_{Heat}}
\end{gather}
, as $E_{Heat}$ and $E_{Ion}$ are Gaussian distributed, but in the case of $Q$
there is bias between $\overline{Q}$ and $\mean{Q}$: 
\begin{gather}
\mean{Q(E_{EIon},E_{Recoil})} = \mean{\exp \left( \begin{pmatrix} E_{Ion} - \mean{E_{Ion}} \\ E_{Heat} - \mean{E_{Heat}} \end{pmatrix} \nabla \right) Q(E_{Ion},E_{Recoil})} \\
= \mean{Q(\mean{E_{Ion}},\mean{E_{Heat}}) + \frac{\partial Q}{\partial E_{Ion}} \frac{\partial Q}{E_{Heat}} (E_{Ion} - \mean{E_{Ion}})(E_{Heat} - \mean{E_{Heat}}) + \right. \\ \left. \frac{1}{2} \frac{\partial^2 Q}{\partial E_{Ion}^2} (E_{Ion} - \mean{E_{Ion}})^2 + \frac{1}{2} \frac{\partial^2 Q}{\partial E_{Heat}^2} (E_{Heat} - \mean{E_{Heat}})^2 + \mathcal O \left( \begin{pmatrix} E_{Ion} - \mean{E_{Ion}} \\ E_{Heat} - \mean{E_{Heat}} \end{pmatrix}^3 \right)} \\
= \overline{Q} + \left( \frac{1}{\mean{E_{Recoil}}} + \frac{V}{\epsilon_\gamma} \frac{\mean{E_{Ion}}}{\mean{E_{Recoil}}^2} \right) \left( 1 + \frac{V}{\epsilon_\gamma} \right) \frac{\mean{E_{Ion}}}{\mean{E_{Recoil}}^2} \sigma_{Ion-Heat}^2 \\ +  \left( \frac{V}{\epsilon_\gamma} \frac{1}{\mean{E_{Recoil}}^2} \right) \left( 1 +  \frac{V}{\epsilon_\gamma} \frac{\mean{E_{Ion}}}{\mean{E_{Recoil}}} \right) \sigma_{Ion}^2 + \left(1 + \frac{V}{\epsilon_\gamma} \right)^2 \frac{\mean{E_{Ion}}}{\mean{E_{Recoil}}^3} \sigma_{Heat}^2 + \mathcal O \left(\mean{ \begin{pmatrix} E_{Ion} - \mean{E_{Ion}} \\ E_{Heat} - \mean{E_{Heat}} \end{pmatrix}^3} \right)
\end{gather}
If the covariance matrix of $\begin{pmatrix} E_{Ion} \\ E_{Heat} \end{pmatrix}$ has very small entries, the square terms can be neglected.
Additionally "`mchist"' is fitted with the pdf "`fkt"' alias 'f' and "`aFitOption"' and the histogram is stored in a ROOT file:
\begin{quote}
<$E_{Ion}$>\_<$E_{Heat}$>\_<$\sigma_{E_{Ion}}$>\_<$\sigma_{E_{Heat}}$>.root
\end{quote}
\item
\begin{verbatim}
void ShowBinGausDistribution(Int_t aMinNumEntries = 400,
                             Int_t aMaxNumEntries = 1E50)
\end{verbatim}
This method builds the histogram "pdfhist" for the theoretical distribution, where the bin contents $n_{pdf,i}$ are determined by evaluating the fitting function of the monte carlo histogram at the bin center values and a histogram "histres" representing the differences for each bin between the the entries in the monte carlo histogram and the pdf histogram. Then it builds a TH1D "gaushist" and fills it with the $z_i$ calculated from the entries of "mchist" and "`pdfhist"'. Only entries with bin content larger than "aMinNumEntries" and smaller than "aMaxNumEntries" in "pdfhist" are considered. As the histograms might have different numbers of effective entries in the histograms' ranges, the $z_i$ need some correction:
\begin{gather}
z_i = \frac{n_{mc,i} - c \cdot n_{pdf,i}}{n_{mc,i} + c^2 \cdot n_{pdf,i}}
\end{gather} 
where $c = \frac{n_{entries,mc}}{n_{entries,pdf}}$ is the quotient of the effective entries of both histograms.
Then with the number of $z_i$ (number of degrees of freedom) and the sum
\begin{gather}
\chi^2 = \sum_i z_i
\end{gather}
a $\chi^2$ test can be applied.
$\chi^2$, the number of degrees of freedom, the percentage of collected entries
are printed. 
\item
\begin{verbatim}
void MakeGraphs(const Char_t* aFileFormat = "pdf",
                Double_t aSignificanceLevel = 0.01)
\end{verbatim}
This method makes graphs for the monte carlo histograms, their projections on both axis (<parameter\_list>\_px.<aFileFormat>) and <parameter\_list>\_py.<aFileFormat>), the projections of the residual histograms on both axis, ((<parameter\_list>\_pxres.<aFileFormat>) and <parameter\_list>\_respy.<aFileFormat>)
and the Gaus histograms (<parameter\_list>\_gaus.<aFileFormat>) filled with $z_i$ for all ROOT files in the current working directory and saves them in the specified file format. \\
Additionally it makes a tex file "graphics.tex" with includegraphics-commands for all images and a tex file "table.tex" which contains a table showing the results of the $chi^2$ tests based on the specified significance level.
\end{enumerate}