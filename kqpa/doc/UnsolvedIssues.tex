\section{Unsolved issues}
\begin{itemize}
\item At the moment, the likelihood $g(\begin{pmatrix} E_{recoil} \\ Q \end{pmatrix}_{exp} | \begin{pmatrix} E_{recoil} \\ Q \end{pmatrix}_{true})$ \ref{probfunc} \\
(implemented in \begin{verbatim} KErecoilQDensity::SingleEventProbDensity(Double_t* x, Double_t* par) \end{verbatim}) is used as the a-posteriori pdf $h(\begin{pmatrix} E_{recoil} \\ Q \end{pmatrix}_{true} | \begin{pmatrix} E_{recoil} \\ Q \end{pmatrix}_{exp})$
under the assumption of a flat prior $f(\begin{pmatrix} E_{recoil} \\ Q \end{pmatrix}_{true}) = \mbox{const}$ in the KQContourPoint and KQContourPointList class. The covariance matrix $C$ is at the moment evaluated at $\begin{pmatrix} E_{recoil} \\ Q \end{pmatrix}_{exp}$ which is only a good approximation in the vicinity of $\begin{pmatrix} E_{recoil} \\ Q \end{pmatrix}_{exp}$. Actually the covariance matrix in the likelihood as well as the a-posteriori pdf depends on the true value $\begin{pmatrix} E_{recoil} \\ Q \end{pmatrix}_{true}$. The true values for $E_{recoil}$ and $Q$ have to be converted into the true values of $E_{ion}$ and $E_{heat}$ (Neganov Luke) and then the covariance matrix can be determined by interpolating between baseline and calibration level uncertainties for a specific detector/run. \\
\begin{verbatim}  (KQUncertainty::GetChannelUncertainty(...)) \end{verbatim}
The corresponding function
\begin{verbatim} KErecoilQDensity::SingleEventTrueProbDensity(Double_t* x, Double_t* par) \end{verbatim}
is implemented, but not yet integrated in the KQContourPoint and KQContourPointList classes.
Then it might be necessary to store information about baseline and calibration uncertainties instead of uncertainties for single events.

\end{itemize}