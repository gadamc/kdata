\documentclass[10pt, a4paper]{article}
\usepackage[english]{babel}
\usepackage[utf8]{inputenc}
\usepackage[T1]{fontenc}
\newcommand{\changefont}[3]{
\fontfamily{#1} \fontseries{#2} \fontshape{#3} \selectfont}
\usepackage{palatino}
\usepackage{wrapfig}
\usepackage{caption}
\usepackage{graphicx}
%\usepackage{dsfont}
\usepackage[fleqn]{amsmath}
\usepackage{amssymb}
\usepackage{url}
\usepackage{color}
\usepackage{colortbl}
\usepackage{lscape}
\usepackage{float}
%\usepackage{booktabs}
\usepackage{multirow}
\usepackage{multicol}
\usepackage[colorlinks, pdfpagelabels, pdfstartview={FitH}, pdfpagelayout={OneColumn}, bookmarksopen=true, bookmarksopenlevel=1, bookmarksnumbered=true, linkcolor=black, hyperindex, plainpages=false, hypertexnames=false, citecolor=black]{hyperref}
%\usepackage{scrpage2}
\usepackage{geometry}
\geometry{hmargin=2cm,top=2cm,bottom=2cm}


\usepackage{fancyhdr}
\pagestyle{fancy} %eigener Seitenstil
\fancyhf{}
\fancyfoot[C]{\thepage}
\fancyhead[C]{$E_{recoil}$-$Q$-Distribution}
\setlength{\headheight}{18pt}
\renewcommand{\headrulewidth}{0.4pt}
\renewcommand{\footrulewidth}{0.4pt}

\newcounter{asdf}

\renewcommand{\labelenumi}{\alph{enumi}) \hspace{1cm}}
%\setlength{\textwidth}{10cm}
%\setlength{\textheight}{23cm}
%\setlength{\oddsidemargin}{-0.5cm}
%\setlength{\topmargin}{-2cm}

\newcommand{\bracket}[3]{\left \langle #1 \middle | #2 \middle | #3 \right \rangle}
\newcommand{\braket}[2]{\left \langle #1 \middle | #2 \right \rangle}
\newcommand{\ket}[1]{\left| #1 \right \rangle}
\newcommand{\bra}[1]{\left \langle #1 \right|}

\title{$E_{recoil}$-$Q$-Distribution}
\author{Daniel Wegner}

\begin{document}
\maketitle
\tableofcontents
\newpage

\section{Derivation}
The aim of this paper is to derive the propability density of the quantity
\begin{gather}
\vec{q} = \begin{pmatrix} E_{recoil} \\ Q \end{pmatrix}
\end{gather}
The quantity $\vec{q}$ depends from 
\begin{gather}
\vec{r} = \begin{pmatrix} E_{ion} \\ E_{heat} \end{pmatrix}
\end{gather}
The dependency is given by
\begin{gather}
T: \vec{r} \rightarrow \vec{q} , \begin{pmatrix} E_{ion} \\ E_{heat} \end{pmatrix}  \rightarrow \begin{pmatrix} E_{recoil} \\ Q \end{pmatrix} = \begin{pmatrix} \left( 1 + \frac{V}{\epsilon_\gamma} \right) E_{heat} - \frac{V}{\epsilon_\gamma} E_{ion} \\ \frac{E_{ion}}{\left( 1 + \frac{V}{\epsilon_\gamma} \right) E_{heat} - \frac{V}{\epsilon_\gamma} E_{ion}} \end{pmatrix}
\end{gather}
Then the inverse transformation is given by
\begin{gather}
T^{-1}: \vec{q} \rightarrow \vec{r}, \begin{pmatrix} E_{recoil} \\ Q \end{pmatrix} \rightarrow \begin{pmatrix} E_{ion} \\ E_{heat} \end{pmatrix} = \begin{pmatrix} Q E_{recoil} \\ \frac{1 + Q \frac{V}{\epsilon_\gamma}}{1 + \frac{V}{\epsilon_\gamma}} E_{recoil} \end{pmatrix}
\end{gather}
The derivative then is given by
\begin{gather}
T'(\vec{r}) = \begin{pmatrix} \frac{\partial E_{recoil}}{\partial E_{ion}} & \frac{\partial E_{recoil}}{\partial E_{heat}} \\ \frac{\partial Q}{\partial E_{ion}} & \frac{\partial Q}{\partial E_{heat}} \end{pmatrix}
= \begin{pmatrix} - \frac{V}{\epsilon_\gamma} & 1 + \frac{V}{\epsilon_\gamma} \\ \frac{1}{E_{recoil}} + \frac{V}{\epsilon_\gamma} \frac{E_{ion}}{E_{recoil}^2} & - \left( 1 + \frac{V}{\epsilon_\gamma} \right) \frac{E_{ion}}{E_{recoil}^2} \end{pmatrix}
\end{gather}
and thus the determinant of it is
\begin{gather}
\det T'(\vec{r}) =  \frac{V}{\epsilon_\gamma} \left( 1 + \frac{V}{\epsilon_\gamma} \right) \frac{E_{ion}}{E_{recoil}^2} - \left(1 + \frac{V}{\epsilon_\gamma}\right) \left( \frac{1}{E_{recoil}} + \frac{V}{\epsilon_\gamma} \frac{E_{ion}}{E_{recoil}^2} \right) = - \frac{1 + \frac{V}{\epsilon_\gamma}}{E_{recoil}}
\end{gather}
Obviously for all values for $E_{recoil}$ the determinant doesn't vanish and thus the transformation is invertible in the whole domain according to the inverse function theorem.

If we assume that $\vec{r}$ is a multivariate normal-distributed quantity, that means it follows the propability density
\begin{gather}
f(\vec{r}) = \frac{1}{2 \pi \sqrt{\det C}} \exp \left( - \frac{1}{2} (\vec{r} - \vec{r}_0)^T C^{-1} (\vec{r} - \vec{r}_0) \right)
\end{gather}
with the covariance matrix
\begin{gather}
C = \begin{pmatrix} \sigma_{ion}^2 & \sigma_{ion-heat}^2 \\ \sigma_{ion-heat}^2 & \sigma_{heat}^2 \end{pmatrix}
\end{gather}
, we get the propability density function $g(\vec{q})$ \cite{Henze}[p. 246]
\begin{gather}
g(\vec{q}) = \frac{f(T^{-1}(\vec{q}))}{\left| det T'(T^{-1}(\vec{q})) \right|} \\ =  \exp \left( - \frac{1}{2}  \begin{pmatrix} Q E_{recoil} - \overline{E_{ion}}\\ \frac{1 + Q \frac{V}{\epsilon_\gamma}}{1 + \frac{V}{\epsilon_\gamma}} E_{recoil} - \overline{E_{heat}} \end{pmatrix}^T \begin{pmatrix} \sigma_{ion}^2 & \sigma_{ion-heat}^2 \\ \sigma_{ion-heat}^2 & \sigma_{heat}^2 \end{pmatrix}^{-1} \begin{pmatrix} Q E_{recoil} - \overline{E_{ion}} \\ \frac{1 + Q \frac{V}{\epsilon_\gamma}}{1 + \frac{V}{\epsilon_\gamma}} E_{recoil} - \overline{E_{heat}} \end{pmatrix} \right) \\
\times  \frac{\left| E_{recoil} \right| }{2 \pi \sqrt{\sigma_{ion}^2 \sigma_{heat}^2 - \sigma_{ion-heat}^4} \left(1 + \frac{V}{\epsilon_\gamma}\right)} \label{propfunc}
\end{gather} 
\section{Statistical moments}
In order to determine the means $\left \langle Q \right \rangle$ and $\left \langle E_{recoil} \right \rangle$ we have to calculate
\begin{gather}
\left \langle Q \right \rangle = \int_{-\infty}^{\infty} dQ \, Q \cdot  \int_{-\infty}^{\infty} dE_{recoil} \, g(E_{recoil},Q) \\
\left \langle E_{recoil} \right \rangle = \int_{-\infty}^{\infty} dQ \int_{-\infty}^{\infty} dE_{recoil} E_{recoil} g(E_{recoil},Q)
\end{gather}
Obviously the exponent in $g(E_{recoil},Q)$ is a square polynomial in $Q$ as well as in $E_{recoil}$.
So we can write
\begin{gather}
g(E_{recoil},Q) = k \cdot \left| E_{recoil} \right| \cdot \exp \left( a_{E_{recoil}} E_{recoil}^2 + b_{E_{recoil}} E_{recoil} + c_{E_{recoil}} \right) \\
= k \cdot \left| E_{recoil} \right| \cdot \exp \left( a_Q Q^2 + b_Q Q + c_Q \right)
\end{gather}
with
\begin{gather}
k = \frac{1}{\sqrt{\sigma_{ion}^2 \cdot \sigma_{heat}^2 - \sigma_{ion-heat}^4} \left(1 + \frac{V}{\epsilon_\gamma} \right)} \\
a_{E_{recoil}} = \frac{Q^2 \sigma_{heat}^2}{2 \left( \sigma_{ion}^2 \sigma_{heat}^2 - \sigma_{ion-heat}^4 \right)} \\
b_{E_{recoil}} = \frac{1}{\sigma_{ion}^2 \cdot \sigma_{heat}^2 - \sigma_{ion-heat}^4}
\end{gather}

Then the means are
\begin{gather}
\overline{E_{recoil}} = \\
\overline{Q} = \frac{\left| E_{recoil} \right| }{2 \pi \sqrt{\sigma_{ion}^2 \sigma_{heat}^2 - \sigma_{ion-heat}^4} \left(1 + \frac{V}{\epsilon_\gamma}\right)} \cdot 
\end{gather}



\section{Toy experiments}
In order to test the goodness of this density function, one can do monte carlo simulation.This can be done in ROOT by creating many events $(E_{Recoil},Q)$ from Gaussian distributed quantities $E_{Ion}$ with uncertainty $\sigma_{E_{Ion}}$ and $E_{Heat}$ with uncertainty $\sigma_{E_{Heat}}$ and fixed parameters $V$, $\epsilon$,
where
\begin{gather}
E_{Recoil} = \left( 1 + \frac{V}{\epsilon} \right) E_{Heat} - \frac{V}{\epsilon} E_{Ion} \\
Q = \frac{E_{Ion}}{E_{Recoil}}
\end{gather}
Then by filling a TH2D histogram with these events and compare it to another TH2D histogram created from the propability density function (pdf) $g(E_{Recoil},Q)$ by the TH2D::FillRandom() method, one can do a $\chi^2$ test to test the null hypothesis $H_0$, that both samples origin from the same distribution.
For each bin the quantity
\begin{gather}
z_i = \frac{n_{pdf,i} - n_{mc,i}}{\sqrt{n_{pdf,i}+n_{mc,i}}}
\end{gather}
can be determined which should be standard normally distributed for high numbers of events $n_{pdf,i}$ and $n_{mc,i}$.
The test then is applied on
\begin{gather}
\chi^2 = \sum_i z_i^2
\end{gather}
where the sum goes over all bins with more than a certain number of events, which should be high enough to be normally distributed in good approximation. \\

\subsection{Integral error estimation}
The exact way to determine the expected bin contents would be to integrate $g(E_{recoil,Q}$ over the ranges of the bin and multiply with the total number of entries $n_{entries}$:
\begin{gather}
n_{i,pdf} = n_{entries} \cdot \int_{E_{recoil,i,min}}^{E_{recoil,i,max}} dE_{recoil} \int_{Q_{i,min}}^{Q_{i,max}} dQ \, g(E_{recoil},Q)
\end{gather}
As these integrations are very time-consuming the effort can be reduced by taylor-expanding $g(E_{recoil},Q$ at the centers of the bins.
Then with $\vec{a}_i = \begin{pmatrix} E_{recoil,i,min} \\ Q_{i,min} \end{pmatrix}$ and $\vec{b}_i = \begin{pmatrix} E_{recoil,i,max} \\ Q_{i,max} \end{pmatrix}$ we have
\begin{gather}
n_{i,pdf} = n_{entries} \cdot \int_{E_{recoil,i,min}}^{E_{recoil,i,max}} dE_{recoil} \int_{Q_{i,min}}^{Q_{i,max}} dQ \, g(E_{recoil},Q) \\
 = n_{entries} \cdot \int_{E_{recoil,i,min}}^{E_{recoil,i,max}} dE_{recoil} \int_{Q_{i,min}}^{Q_{i,max}} dQ \sum_{n_{E_{recoil}} = 0}^\infty \sum_{n_{Q} = 0}^\infty
\frac{\left( \vec{x} - \frac{\vec{b}_i + \vec{a}_i}{2} \right)^{n_{E_{recoil}}} \left( \vec{x} - \frac{\vec{b}_i + \vec{a}_i}{2} \right)^{n_{Q}}}{n_{E_{recoil}}! n_Q!}
\cdot \frac{\partial^{n_{E_{recoil}}}}{\partial E_{recoil} ^{n_{E_recoil}}} \frac{\partial^{n_{Q}}}{\partial Q^{n_{Q}}} g(E_{recoil,i},Q_i)
\end{gather}
In the double for all terms where $n_{E_{recoil}}$ and $n_{Q}$ are even the integral vanish as the indefinite integrals are odd with respect to the center of the bin:
\begin{gather}
n_{i,pdf} = n_{entries} \cdot \left(  \left| \vec{b}_i - \vec{a}_i \right| g(E_{recoil,i},Q) \right. \\
+ \frac{1}{2} \frac{\partial^2}{\partial E_{recoil}^2} g(E_{recoil,i},Q_i) \cdot \left( \frac{E_{recoil,i,max} - E_{recoil,i,min}}{2} \right)^3 
+ \frac{1}{2} \frac{\partial^2}{\partial Q^2} g(E_{recoil,i},Q_i) \cdot \left( \frac{Q_{i,max} - Q_{i,min}}{2} \right)^3  \\
+ \left. \mathcal O \left( \left( \frac{E_{recoil,i,max} - E_{recoil,i,min}}{2} \right)^3 \left( \frac{Q_{i,max} - Q_{i,min}}{2} \right)^3 \right) \right)
\end{gather}
This procedure is applied for some examples in the following: \\[0.5cm]
\captionof{figure}{Histograms with Monte Carlo events for some parameter combinations of $\bar{E_{Ion}}$,$\bar{E_{Heat}}$, $\sigma_{E_{Ion}}$, and $\sigma_{E_{Heat}}$ fitted with the pdf $f(E_{Recoil},Q) = c \cdot g(E_{Recoil},Q)$ and distribution of $z_i$ for minimal number of pdf events $n_{min}>400$}


\begin{minipage}{\textwidth}

\captionof{table}{$\chi^2$ values for the corresponding parameter combinations and acceptance of the null hypothesis $H_0$}
\begin{tabular}{|c|c|c|c|c|c|c|c|c|c|} \hline
$\overline{E_{Ion}}$ & $\overline{E_{heat}}$ & $\sigma_{E_{Ion}}$ & $\sigma_{E_{Heat}}$ & $\chi^2$ value & ndf & $n_{min}$ & TMath::Prop($\chi^2$,ndf) & CL of pdf &  $H_0$ \\ \hline \hline
100 & 100 & 1 & 1 & 57179 & 57110 & 400 & 0.418 & 90.1\% &  yes \\ \hline
100 & 100 & 5 & 1 & 34762.5 & 34576 & 400 & 0.239 & 95.3\% & yes \\ \hline
100 & 100 & 1 & 5 & 29530 & 29038 & 400 & 0.020 & 96.2\%  & yes \\ \hline
100 & 50 & 1 & 1 & 65629 & 64877 & 400 & 0.019 & 87.4\%  & yes \\ \hline
100 & 50 & 5 & 1 & 51710 & 51732 & 400 & 0.526 & 91.2\% & yes \\ \hline
100 & 50 & 1 & 5 & 29035 & 29322 & 400 & 0.882 & 96.4\% & yes \\ \hline
50 & 100 & 1 & 1 & 90959 & 90805 & 400 & 0.358 & 68.7\% & yes \\ \hline
50 & 100 & 1 & 5 & 73109 & 73203 & 400 & 0.597 & 83.8\% & yes \\ \hline
50 & 100 & 5 & 1 & 61415 & 61112 & 400 & 0.193 & 88.7\% & yes \\ \hline
20 & 20 & 5 & 5 & 58015 & 58449 & 400 & 0.898 & 84.7\% & yes \\ \hline
20 & 20 & 1 & 5 & 31008 & 30129 & 400 & 0.0002 & 89.3\% & no \\ \hline
20 & 20 & 5 & 1 & 34082 & 33851 & 400 & 0.188 & 86.5\% & yes \\ \hline

\end{tabular} \\[1cm]

\end{minipage}

The confidence level of the propability density function (CL of pdf) gives the percentage of the events in the $\chi^2$ sum to the total sum of all monte carlo events.
For the acceptance of the null hypothesis $H_0$, a significance level of 1\% is assumed. That means it is accepted if TMath::Prob($\chi^2$,ndf)>0.01.
So in one case out of 12 the null hypothesis $H_0$ has to be rejected. 
\subsection{script}
The plots have been created by the script on kalinka in my home directory:
\begin{verbatim}
/kalinka/home/wegner/ERecoilQDistribution/ERecoilQDist_v20.C
\end{verbatim}
This file offers two methods:
\begin{enumerate}
\item \begin{verbatim}
ERecoilQDist_v20(Double_t anEIonMean = 100,
								 Double_t anEHeatMean = 100,
								 Double_t anEIonSigma = 1,
								 Double_t anEHeatSigma = 1,
								 Double_t aNumBinsX = 2000,
								 Double_t aNumBinsY = 2000,
								 Double_t aNumTimes = 1,
								 Long_t aNumEntries = 1E9,
								 Double_t aV = 3,
								 Double_t anEpsilon = 1,
								 Option_t* aFitOption = "`0LI"')
\end{verbatim}
This methods creates two histograms "`mchist"' and "`pdfhist"', the first from creating random numbers distributed according the given parameters, the second from the \begin{verbatim} TH1::FillRandom(const char* fname,Int_t ntimes = 500) \end{verbatim} method according to the theoretical propability density function "`f"'. The histograms have the dimensions (aNumBinsX,aNumBinsY) and the boundaries are chosen, so that it covers $\pm$ aNumSigmas standard deviations calculated from error propagation around the center value
\begin{gather}
\overline{E_{Recoil}} = \left(1 + \frac{V}{\epsilon_\gamma} \right) \overline{E_{Heat}} - \frac{V}{\epsilon_\gamma} \overline{E_{Ion}} \\
\overline{Q} = \frac{\overline{E_{Ion}}}{\overline{E_{Recoil}}} 
\end{gather}
Additionally "`mchist"' is fitted with the pdf "`fkt"' alias 'f' and "`aFitOption"' and both histograms are filled with "`aNumEntries"'. "`mchist"' might have less effective entries, as some generated random numbers might be out of range of the histogram. This procedure is applied "`aNumTimes"' and the histrograms filled in a TTree "`tree"' with branchnames "`mc-hists"' and "`pdf-hists"'. Then the tree is stored in a file (TFile "`file"') of the form:
\begin{quote}
<$E_{Ion}$>\_<$E_{Heat}$>\_<$\sigma_{E_{Ion}}$>\_<$\sigma_{E_{Heat}}$>\_<aNumEntries>.root
\end{quote}
\item
\begin{verbatim}
void ShowBinGausDistribution(const Char_t* aFileName = "`"',
														Int_t anEntry = 0,
														Int_t aMinNumEntries = 3000)
\end{verbatim}
This method reads the entry "`anEntry"' in the tree "`tree"' in the file "`aFileName"', builds a TH1D "`gaushist"' and fills it with the $z_i$ calculated from the entries of "`mchist"' and "`pdfhist"'. Only entries with bin content larger than "`aMinNumEntries"' in "`pdfhist"' are considered. As the histograms might have different numbers of effective entries in the histograms' ranges, the $z_i$ need some correction:
\begin{gather}
z_i = \frac{n_{mc,i} - c \cdot n_{pdf,i}}{n_{mc,i} + c^2 \cdot n_{pdf,i}}
\end{gather} 
where $c = \frac{n_{entries,mc}}{n_{entries,pdf}}$ is the quotient of the effective entries of both histograms.
Then with the number of $z_i$ (number of degrees of freedom) and the sum
\begin{gather}
\chi^2 = \sum_i z_i
\end{gather}
a $\chi^2$ test can be applied.
$\chi^2$, the number of degrees of freedom, the percentage of collected entries 

\end{enumerate}









\begin{thebibliography}{1cm}
\bibitem{Henze} Henze, Stochastik I, Einfuhrung in die Wahrscheinlichkeitstheorie und Statistik, 2004
\end{thebibliography}







\end{document}